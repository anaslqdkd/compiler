\documentclass[a4paper, 12pt]{report}

\usepackage{lmodern} % Police standard sous LaTeX : Latin Modern
% (alternative à la police d'origine développée par Donald
	%Knuth : Computer Modern)
\usepackage[french]{babel} % Pour la langue fran¸caise
\usepackage[utf8]{inputenc} % Pour l'UTF-8
\usepackage[T1]{fontenc} % Pour les césures des caractères accentués
\renewcommand{\thesection}{\Roman{section}}
\usepackage{amssymb}
\usepackage{enumitem}
\renewcommand{\arraystretch}{1.3}
\usepackage{pifont}
\usepackage{hyperref}
\usepackage{xcolor}
\usepackage{stmaryrd}
\usepackage{amsmath,amsfonts, amssymb}
\usepackage{graphicx}
\usepackage{tikz}
\usetikzlibrary{shapes.geometric}
\usetikzlibrary{automata, arrows.meta, positioning}

\begin{document}



   - Le rapport est à rendre la veille du  jour de la soutenance, dans le casier de votre enseignant de TP. Je souhaite une version papier, simplement agrafée, pas de reliure.  Pensez à inclure une partie gestion de projet avec entre autres, vos temps de travail estimés et les missions de chaque membre du groupe.
Prévoyez également une description courte mais pertinente de la structure de votre Table des Symboles.

Concernant les "schémas de traduction" à présenter : il s'agit de mettre en évidence dans votre dossier comment vous avez traduit en assembleur certaines structures "intéressantes" du langage, comme les appels de fonctions, conditionnelles imbriquées, record, etc.
Pensez aussi à faire figurer quelques jeux d'essais pertinents, et le source en Mini Python de votre programme de démonstration.


\end{document}

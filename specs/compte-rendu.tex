\documentclass[a4paper, 12pt]{report}

\usepackage{lmodern} % Police standard sous LaTeX : Latin Modern
% (alternative à la police d'origine développée par Donald
	%Knuth : Computer Modern)
\usepackage[french]{babel} % Pour la langue fran¸caise
\usepackage[utf8]{inputenc} % Pour l'UTF-8
\usepackage[T1]{fontenc} % Pour les césures des caractères accentués
\renewcommand{\thesection}{\Roman{section}}
\usepackage{amssymb}
\usepackage{enumitem}
\renewcommand{\arraystretch}{1.3}
\usepackage{pifont}
\usepackage{hyperref}
\usepackage{xcolor}
\usepackage{stmaryrd}
\usepackage{amsmath,amsfonts, amssymb}
\usepackage{graphicx}
\usepackage{tikz}
\usetikzlibrary{shapes.geometric}
\usetikzlibrary{automata, arrows.meta, positioning}

\newcommand{\diagr}[1]{\begin{center}\begin{tikzpicture}[node distance = 2cm, on grid, auto]
#1
\end{tikzpicture}\end{center}}

\begin{document}
\tikzset{square state/.style={draw,regular polygon,regular polygon sides=4}}

\chapter{Conception du lexeur}

\section{Conventions de notation}

\subsection{Types de \textit{Token}}

Dans cette partie, on désignera par $i$ (avec $i \in \mathbb{N}^*$) le numéro de la ligne où le \textit{Token} a été lu.\\

\begin{enumerate}

\item Retours chariot \& délimitations de blocs
\begin{itemize}
\item Token indiquant un saut de ligne : \textbf{Token(NEWLINE, $i$)} ;
\item Token signalant le début d'une indentation : \textbf{Token(BEGIN, $i$)} ;
\item Token marquant la fin d'une indentation : \textbf{Token(END, $i$)} ;
\item Token indiquant la fin du fichier source : \textbf{Token(EOF, $i$)}.\\
\end{itemize}

\item Identificateur, représentant une variable, une fonction ou un paramètre : \textbf{Token(IDENTIFIER, <identificateur>, $i$)}.\\

\item Mot-clé : \textbf{Token(KEYWORD, <mot clé>, $i$)}. On rappelle que les mots-clé, réservés par le langage et ne pouvant pas être utilisées comme identifiants, sont les suivants : \textit{and}, \textit{or}, \textit{if}, \textit{else}, \textit{for}, \textit{in}, \textit{not}, \textit{True}, \textit{False}, \textit{print}, \textit{def}, \textit{return} et \textit{None}.\\

\item Opérateurs binaires (utilisés pour faire des opérations binaires) :
\begin{itemize}
\item \textbf{Token(PLUS, $i$)} ;
\item \textbf{Token(MINUS, $i$)} ;
\item \textbf{Token(MULTIPLY, $i$)} ;
\item \textbf{Token(FLOOR{\_}DIVIDE, $i$)} ;
\item \textbf{Token(MODULO, $i$)}\footnote{On considère l'opérateur "\textit{modulo}" comme étant l'opérateur binaire qui associe à deux entiers naturels le reste de la division euclidienne du premier par le second.} ;
\item \textbf{Token(LESS{\_}EQUAL, $i$)} ;
\item \textbf{Token(GREATER{\_}EQUAL, $i$)} ;
\item \textbf{Token(LESS, $i$)} ;
\item \textbf{Token(GREATER, $i$)} ;
\item \textbf{Token(NOT{\_}EQUAL, $i$)} ;
\item \textbf{Token(EQUAL, $i$)} ;
\item \textbf{Token(AND, $i$)} ;
\item \textbf{Token(OR, $i$)}.\\
\end{itemize}

\item Opérateurs unaires (c'est-à-dire utilisés pour faire des opérations unaires) : \textbf{Token(UNARY{\_}MINUS, $i$)} et \textbf{Token(NOT, $i$)}.\\

\item Opérateur d'assignation ("$:=$") : \textbf{Token(ASSIGNMENT, $i$)}.\\

\item Types pris en charge par le langage : \textbf{Token(INTEGER, <int>, $i$)} et \textbf{Token(STRING, <str>, $i$)}.\\

\item Délimiteurs :
\begin{itemize}
\item Crochet ouvrant ("[") : \textbf{Token(LBRACKET,$i$)} ;
\item Crochet fermant ("]") : \textbf{Token(RBRACKET,$i$)} ;
\item Parenthèse ouvrante ("(") : \textbf{Token(LPAREN,$i$)} ;
\item Parenthèse fermante (")") : \textbf{Token(RPAREN,$i$)} ;
\item Virgule : \textbf{Token(COMMA,$i$)} ;
\item Point-virgule : \textbf{Token(COLON,$i$)}.\\
\end{itemize}

\end{enumerate}

\subsection{Ensembles}

Pour alléger les notations dans les automates qui suivent, on propose les notations suivantes :
\begin{itemize}[label=-]

\item $\alpha$ désignera tout caractère alphanumérique (minuscule ou majuscule) :
\[ \alpha = [a-z] \ | \ [A-Z]. \]

\item $\mathcal{D}$ désignera tout chiffre :
\[ \mathcal{D} = [0-9]	. \]

\item $\mathcal{O}$ désignera tout opérateur formé d'un seul symbole :
\[ \mathcal{O} = [+ \ | \ - \ | \ * \ | \ \% ]	. \]

\item $\mathcal{A}$ désignera n'importe quel symbole reconnu par la grammaire.

\end{itemize}

\newpage

\section{Sous-automates}

\subsection{Identificateurs \& mots-clés}

\begin{center}\begin{tikzpicture}[node distance = 4cm, on grid, auto]
\node (q0) [state,accepting,initial,initial text = {}] {$q_0$};
\node (q1)  [state,right = of q0] {$q_1$};
\node  (q2) [state,accepting,right = of q1] {$q_2$};
\node(q2d)[below=1cm of q2] {<IDENTIFIER,value>};
\node[below=0.5cm of q2d] {ou <KEYWORD,value>};

\path [-stealth, thick]
	(q0) edge node {$\alpha \ | \ \_$} (q1)
	(q1) edge node {$\wedge(\alpha \ | \ \_ \ | \ \mathcal{D})$} (q2)
	(q1) edge [loop above]  node {$\alpha \ | \ \_ \ | \ \mathcal{D}$}();
\end{tikzpicture}\end{center}

\subsection{Délimiteurs}

\begin{center}\begin{tikzpicture}[node distance = 6cm, on grid, auto]
\node (q0) [state,accepting,initial,initial text = {}] {$q_0$};
\node (q1) [state,accepting,right = of q0] {$q_1$};
\node(q1d)[below=1cm of q1] {<value>};

\path [-stealth, thick]
	(q0) edge node {$[ [ \ | \ ] \ | \ ( \ | \ ) \ | \ , \ | \ ; ]$} (q1);
\end{tikzpicture}\end{center}

\subsection{Chaînes de caractère}

\begin{center}\begin{tikzpicture}[node distance = 3.5cm, on grid, auto]
\node (q0) [state,accepting,initial,initial text = {}] {$q_0$};
\node (q1)  [state,right = of q0] {$q_1$};
\node (q2) [state,accepting,right = of q1] {$q_2$};
\node (q3) [state, below = of q1] {$q_3$};
\node(q2d)[below=1cm of q2] {<STRING,value>};

\path [-stealth, thick]
	(q0) edge node {$"$} (q1)
	(q1) edge node {$"$} (q2)
	(q1) edge [loop above]  node {$\wedge [" \ | \ \backslash]$}()
	(q1) edge [bend left = 20] node {$\backslash$} (q3)
	(q3) edge [bend left = 20] node {$\mathcal{A}$} (q1);
\end{tikzpicture}\end{center}

\subsection{Entiers}

\begin{center}\begin{tikzpicture}[node distance = 4cm, on grid, auto]
\node (q0) [state,accepting,initial,initial text = {}] {$q_0$};
\node (q1)  [state,right = of q0] {$q_1$};
\node  (q2) [state,accepting,right = of q1] {$q_2$};
\node(q2d)[below=1cm of q2] {<INTEGER,value>};

\path [-stealth, thick]
	(q0) edge node {$\mathcal{D}$} (q1)
	(q1) edge node {$\wedge(\mathcal{D})$} (q2)
	(q1) edge [loop above]  node {$\mathcal{D}$}();
\end{tikzpicture}\end{center}

\subsection{Opérateurs unaires}

\begin{center}\begin{tikzpicture}[node distance = 4cm, on grid, auto]
\node (q0) [state,accepting,initial,initial text = {}] {$q_0$};
\node (q2) [state,right = of q0] {$q_2$};
\node (q3) [state,right = of q2] {$q_3$};
\node (q4) [state,accepting,right = of q3] {$q_4$};
\node (q5) [state,accepting,below = of q4] {$q_5$};
\node (q1) [state,accepting,above = of q3] {$q_1$};
\node(q1d)[below=1cm of q1] {<UNARY{\_}MINUS>};
\node(q4d)[below=1cm of q4] {<NOT>};
\node(q5d)[below=1cm of q5] {\textit{raise error}};

\path [-stealth, thick]
	(q0) edge [bend left = 25] node {$-$} (q1)
	(q0) edge node {$n$} (q2)
	(q2) edge node {$o$} (q3)
	(q3) edge node {$t$} (q4)
	(q2) edge [bend right = 25] node {$\wedge[o]$} (q5)
	(q3) edge node {$\wedge[t]$} (q5)	;
\end{tikzpicture}\end{center}

\subsection{Opérateurs binaires}

\begin{center}\begin{tikzpicture}[node distance = 3.5cm, on grid, auto]
\node (q8) [state,accepting] {$q_2$};
\node (q9) [state,accepting,below = of q8] {$q_3$};
\node (q11) [state,accepting,below = of q9] {$q_5$};
\node (q12) [state,accepting,below = of q11] {$q_6$};
\node (q14) [state,accepting,below = of q12] {$q_8$};
\node (q15) [state,accepting,below = of q14] {$q_9$};
\node[below=1cm of q8] {<value>};
\node[below=1cm of q9] {<value>};
\node[below=1cm of q11] {<NOT{\_}EQUAL>};
\node[below=1cm of q12] {\textit{raise error}};
\node[below=1cm of q14] {<FLOOR{\_}DIVIDE>};
\node[below=1cm of q15] {\textit{raise error}};

\node (q7) [state, left = of q9] {$q_1$};
\node (q10) [state, left = of q12] {$q_4$};
\node (q13) [state, left = of q14] {$q_7$};
\node (q6) [state,accepting,initial, left = of q10,initial text = {}] {$q_0$};
\node (q16) [state,accepting,below = of q6] {$q_{10}$};
\node[below=1cm of q16] {<value>};

\path [-stealth, thick]
	(q6) edge node {$(< \ | \ = \ | \ : \ | \ >)$} (q7)
	(q6) edge node {$!$} (q10)
	(q6) edge node {$/$} (q13)
	(q6) edge node {$\mathcal{O}$} (q16)
	
	(q7) edge [bend left = 35] node {$=$} (q8)
	(q7) edge node {$\wedge[=]$} (q9)
	(q10) edge [bend left = 35] node {$=$} (q11)
	(q10) edge node {$\wedge[=]$} (q12)
	(q13) edge node {$/$} (q14)
	(q13) edge [bend right = 35] node {$\wedge[/]$} (q15);
\end{tikzpicture}\end{center}

\newpage

\section{Automate fini déterministe}

Dans cette section, on va détailler l'automate fini déterministe correspondant au lexeur du "Mini-Python". Ce dernier implémentera les sous-automates détaillés précédemment, qui seront tous appelés simultanément à l'état 1 (représenté rectangulaire) de l'automate ci-dessous. Ainsi, n'importe quelle unité lexicale sera reconnue et verra son \textit{Token} ajouté tant qu'aucun des autres symboles "$\backslash n$", "$\#$" ou un caractère non-accepté par le langage.\\

\begin{center}\begin{tikzpicture}[node distance = 5cm, on grid, auto]
\node (q0) [state,initial,initial text = {}] {$q_0$};
\node (q1) [square state,right = of q0] {$q_1$};
\node (q2) [state,right = of q1] {$q_2$};
\node (q3) [state,accepting,above = of q2] {$q_3$};
\node (q4) [state,below = of q2] {$q_4$};
\node (q5) [state,below = of q0] {$q_5$};
\node (q6) [state,accepting,right = of q5] {$q_6$};
\node(q4d)[below=1cm of q4] {<NEWLINE>};
\node(q5d)[below=1cm of q5] {<EOF>};
\node(q3d)[below=1cm of q3] {\textit{raise error}};

\path [-stealth, thick]
	(q0) edge node {} (q1)
	(q1) edge [loop above]  node {\textit{space}}()
	(q1) edge [bend left = 10] node {$\#$} (q2)
	(q2) edge [bend left = 10] node {$\backslash n$} (q1)
	(q2) edge [loop above]  node {$\wedge (\backslash n)$}()
	(q1) edge node {$\wedge (\mathcal{A})$} (q3)
	(q1) edge node {$\backslash n$} (q4)
	(q4) edge [bend left = 25] node {} (q1)
	(q1) edge node {\textit{eof}} (q5)
	(q5) edge node {} (q6)
	;
\end{tikzpicture}\end{center}

\chapter{Conception du parseur}

\section{Grammaire LL(2)}

On propose la grammaire suivante, obtenue à partir de celle-ci dessus, dont les règles figurent en annexe (document "grammar.pdf") :
\[ G = ( \mathcal{N}, \mathcal{T}, \to, S), \quad \text{où}\]
\[  \mathcal{N} = \lbrace S,S',S'',A,B,B',C,C',D,D',E,E',E'',E''',F,G,G',H,H',I,I' \rbrace \]
\[ \text{et } \mathcal{T} = \alpha \cup \mathcal{D} \cup \lbrace +;-;*;/;\%;(;[;);];\backslash, ",:,!,=\rbrace.\]

\section{Élagage de l'arbre}

\subsection{Suppression des éléments lexicaux non-essentiels}

De nombreux \textit{tokens} dans l'arbre à sa sortie du parseur sont purement syntaxiques et n'affectent pas réellement la logique sous-jacente. Il s'agit des \textit{tokens} du types "\textbf{Newline}", "\textbf{def}" et "\textbf{EOF}", ainsi que les virgules et les deux-points. On va donc les éliminer en premier.\\

\subsection{Modification des tuples et des listes}

La prochaine étape que l'on propose est de s'occuper de reformatter les tuples et les listes du Mini-Python. Pour se simplifier la vie, on propose d'appliquer un algorithme récursif qui applique un formatage à chaque fois qu'il voit des n{\oe}uds de l'arbre entourés de \textit{Tokens} de parenthèses / crochets ouvrant(e) et fermant(e).\\

Le formatage adopté est le suivant (on l'explique ici pour un tuple, mais le formatage est le même pour les listes) : dans la liste des enfants, à la place de l'ensemble "n{\oe}ud parenthèse ouvrante + contenu du tuple + n{\oe}ud parenthèse fermante, on place un nouveau n{\oe}ud "tuple", dont les enfants sont le contenu du tuple.\\

Le problème avec un tel algorithme est le recopiage du contenu du tuple. En effet, même si les virgules séparant les différents éléments ont été supprimées à l'étape précédente de l'élagage, il reste tous les non-terminaux qui ont servi à la construction du tuple, or on aimerait que les enfants du n{\oe}ud "tuple" soient directement le contenu du tuple (dans l'ordre). Il faut donc retirer ces non-terminaux (et seulement ceux-là, les autres peuvent encore servir à l'élagage). Dans notre grammaire, il s'agit des non-terminaux $I$, $I_1$, $E_1$ et $E_2$ (on peut noter que $I$ et $I_1$ ne peuvent apparaître que dans les tuples).\\

\subsection{Modification des fonctions}

Pour les fonctions, on détecte leur présence dans l'arbre tout naturellement avec la présence du mot-clé "\textbf{def}". Dès lors, on le retire (car il n'est plus vraiment utile à l'analyse sémantique après cette étape), et on range dans un nouveau n{\oe}ud "\textbf{function}" dans l'ordre l'identifiant contenant le nom de la fonction, un tuple (éventuellement vide) correspondant aux paramètres de la fonction, et un non-terminal contenant en enfants les lignes de code de la fonction.\\

\subsection{Modification des opérateurs}

Aussi, pour la plupart des opérateurs (somme, différences, opérateurs de comparaison, ...), leur arité est de 2. On propose donc de mettre leurs deux paramètres en enfant du n{\oe}ud opérateur (on adopte en quelques sortes une notation préfixée pour ces opérateurs, à l'échelle de l'arbre).\\

\subsection{Modifications des \textbf{print} et des \textbf{return}}

Pour ce qui est des \textbf{print} et des \textbf{return}, le formalisme est le même : on les considère comme des opérateurs unaires, dont la valeur utilisée (à afficher ou à renvoyer) est passée en enfant du n{\oe}ud \textbf{print} / \textbf{return}.\\

\subsection{Modification des \textbf{for} et des \textbf{if}}

À l'instar des fonctions, on a rassemblé les éléments caractéristiques des \textbf{for} en enfants du n{\oe}ud "\textbf{for}". Pour une boucle de la forme "for $i$ in $L$", on stockera dans les enfants du n{\oe}ud "\textbf{for}" dans l'ordre l'identifiant contenant $i$, l'identifiant contenant $L$, et enfin le bloc à exécuter dans la boucle.\\

Pour ce qui est des \textbf{if}, on procède de même, en stockant dans les enfants du n{\oe}ud \textbf{if} dans l'ordre : la condition à vérifier pour rentrer dans le \textbf{if}, le code a exécuter si la condition est vérifiée, et éventuellement un n{\oe}ud "\textbf{else}" contenant le code a exécuter si cela n'est pas le cas.\\

\subsection{Finalisations de l'AST}

Enfin pour conclure l'élagage, on rassemble les chaînes "inutiles" de n{\oe}uds non-terminaux, de sorte que tous les n{\oe}uds représentant des lignes d'un même bloc de code soient tous les enfants du même n{\oe}ud (généralement un non-terminal, comme $S$, $A$ ou $D$).\\

\end{document}
